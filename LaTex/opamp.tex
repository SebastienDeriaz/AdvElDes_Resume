\documentclass[resume]{subfiles}


\begin{document}
\begin{multicols}{3}
\section{Amplificateurs opérationnels}

\subsection{Modèle petits signaux du BJT}
\begin{center}
\begin{tabular}{ll}
$I_0$ & Courant de polarisation sur la sortie\\
$U_{\text{early}}$ & Tension de Early (\SIrange{15}{150}{\volt})\\
$I_B$ & Courant de polarisation de base\\
$U_T$ & Tension thermique (\SI{\approx 25}{\milli\volt})\\
$\beta$ & Gain du transistor\\
$C_m$ & Capacité de Miller (sortie)\\
$C_e$ & Capacité de Miller reportée sur la base
\end{tabular}
\end{center}
$$U_T=\frac{kT}{e}\quad k=\num{1.381e-23}\quad e=\num{1.602e-19}$$
\columnbreak
\subsubsection{Modèle du livre}
\begin{figure}[H]
\centering
\includegraphics[scale=1,page=9]{../KiCad/resume-crop.pdf}
\end{figure}


$$R_{be}=\frac{\beta}{g_m}=\frac{U_T}{I_0}$$

$$I_0=\beta I_B$$
$$g_m=\frac{I_0}{U_T}=\frac{1}{R_E'}$$
$$C_e=(1+\abs{A})C_m\qquad \text{approximation : } C_e\approx AC_m$$
$$A=-\frac{R_c\beta}{R_{be}}$$
$$R_C=\frac{U_{\text{early}}}{I_0}$$


$$f_c=\frac{1}{2\pi R_{be} \abs{A} C_m}$$
A noter que le gain $A$ est le gain vu par le transistor. Si on modifie la suite du circuit pour diminuer le gain, il faudra mettre à jour les valeurs calculées.
\subsubsection{GBW}
Produit constant sur la droite du GBW
$$A\cdot \omega_c = \text{GBW}\qquad \omega_c=2\pi f_c$$
Si on a une application avec $\omega_a$, alors le gain maximal est donné par
$$A_{max_\omega}=\frac{\text{GBW}}{\omega}$$
\subsubsection{Modèle du cours}
\begin{figure}[H]
\centering
\includegraphics[scale=1,page=10]{../KiCad/resume-crop.pdf}
\end{figure}
$$A=-\frac{R_c}{R'_E}$$
$$R'_E=\frac{R_{be}}{\beta}$$

$$f_c=\frac{1}{2\pi \beta R'_E \abs{A} C_m}$$
on suppose que le courant de base est nul (que $R'_E$ est parcouru par le courant du collecteur uniquement).
\subsection{Comportement en fréquence d'un ampli-op}
Le gain est de la forme
$$\boxed{\frac{AU_D}{1+\frac{s}{\omega_0}}}$$
Avec $U_D$ la différence de tension entre les bornes + et -

\end{multicols}
\end{document}